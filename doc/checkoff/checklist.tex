% project checkoff checklist

\documentclass{article}

\usepackage{fullpage}
\usepackage{graphicx}
\usepackage{mathtools}
\usepackage{amssymb}

\title{6.111 Final Project\\Project Checklist}
\date{21 November 2011}
\author{Logan P. Williams\\Jos\'{e} E. Cruz Serrall\'{e}s}

\renewcommand{\labelitemii}{$\Box$}

\begin{document}
\maketitle

\begin{itemize}
% Logan
\item[] {\tt projective\_transform}: processes a stream of incoming pixels, skewing, rotation, and scaling the image by generating new $(x,y)$ coordinates for each pixel corresponding to the four corners of the frame. (Logan)
	\begin{itemize}
	\item Correctly calculates distances and iterator incrementors, using the {\tt sqrt} and {\tt divide} submodules. 
	\item Sends a signal to {\tt LPF} to request new data when initial frame calculations have been done.
	\item Read one new pixel per clock cycle from {\tt LPF}.
	\item Generate one new set of coordinates per clock cycle and transmit to {\tt memory\_interface}.
	\item Pipelines square root and division calculations so that there is no delay for each new line.
	\item Pauses appropriately when {\tt memory\_interface} cannot handle new data.
	\item Can handle ``unexpected'' new frame events.
	\end{itemize}

% Logan
\item[] {\tt object\_recognition}: average the $(x,y)$ tuples for each pixel that matches one of four Cr/Cb regions of interest. (Logan)
	\begin{itemize}
	\item Sums the coordinates of each color that it receives.
	\item Correctly averages each coordinate.
	\item Outputs the list of coordinates and a flag immediately after {\tt ntsc\_capture} has finished processing a frame and the {\tt divide} submodules have finished their averaging operations.
	\item Output ``fake'' downsampling coefficients based on linear estimates of distance.
	\item (Time permitting:) Generate and output $M_x$ and $M_y$ downsampling coefficients after a frame has been captured.
	\end{itemize}

% Jose
\item[] {\tt memory\_interface}: efficiently interfaces with the memory and all of the modules that have to write to and read from ZBT memory. (Jos\'{e})
	\begin{itemize}
	\item Writes to memory data from {\tt ntsc\_capture}.
	\item Reads from memory an image to {\tt vga\_display}.
	\item Outputs to and captures data from {\tt LPF}.
	\item Captures data from {\tt projective\_transform}.
	\item Shifts data locations when {\tt ntsc\_capture} starts providing a new image.
	\item (Time permitting:) Reads an image from flash memory and stores it in RAM for use as the transformed image.
	\end{itemize}

% Jose
\item[] {\tt LPF:} applies lowpass filters, vertically and horizontally, on the image that is to be warped, in order to prevent aliasing at the output. (Jos\'{e})
	\begin{itemize}
	\item (Out of time:) Just fetches pixels from memory and feeds them to {\tt projective\_transform}. LPF does not filtering.
	\item Loads appropriate filter coefficients based on the coefficients $M_x$ and $M_y$ from {\tt object\_recognition}.
	\item Reads data from memory vertically and horizontally, and has the necessary data for the calculation of each output pixel in its buffers.
	\item Mirrors the data appropriately in its buffers when processing near an edge.
	\item Outputs to memory\_interface a pair of pixels that correspond to the convolution sum of the corresponding data.
	\end{itemize}

% Logan
\item[] {\tt ntsc\_capture}: process the incoming video stream and send pixels in sets of two to {\tt memory\_interface} (Logan)
	\begin{itemize}
	\item Capable of reading the incoming video stream from the video ADC.
	\item Can transmit pixels to {\tt vga\_display} for immediate display.
	\item Lumps pixels into groups of two to transmit to {\tt memory\_interface}.
	\item Recognizes pixels matching specific regions of the Cr/Cb plane, and transmits that information to {\tt object\_recognition}.
	\end{itemize}

% Jose
\item[] {\tt vga\_display}: fetches data from memory and displays it on the screen. (Jos\'{e})
	\begin{itemize}
	\item Displays a predefined pattern on the screen.
	\item Requests a pixel one video clock cycle before it is needed.
	\item Reads an image from memory, through {\tt memory\_interface}, and correctly displays it on the screen.
	\end{itemize}

\end{itemize}

\end{document}
