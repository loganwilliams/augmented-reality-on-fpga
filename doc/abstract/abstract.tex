\documentclass[11pt]{article}
\usepackage{geometry}
\geometry{letterpaper}   
\usepackage{setspace}
\onehalfspacing
\usepackage{graphicx}
\usepackage{amssymb}
\usepackage{epstopdf}
\DeclareGraphicsRule{.tif}{png}{.png}{`convert #1 `dirname #1`/`basename #1 .tif`.png}

\title{Recursive Augmented Reality}
\author{Jos\'{e} E. Cruz Serrall\'{e}s \& Logan Williams}
\date{} 

\begin{document}
\maketitle


In our project, we will implement an augmented reality system that can overlay a digital image on video of a real world environment. We begin by reading NTSC video from a video camera and storing it in ZBT SRAM. A picture frame with colored markers on the corners is held in front of the camera. We then perform chroma-based object recognition to locate the co-ordinates of the corners. Using these co-ordinates, we apply appropriate translation, scaling, rotation, and anti-aliasing FIR filters to fit the image to the boundary of the frame. If time allows, we will use a non-linear projective transformation to correct for perspective. We then output VGA video of the original captured image, with the processed image overlayed on top of the frame. The overlayed image (the ``augmentation'') can be arbitrary. When this image is the frame of video that was previously displayed, we call the system ``recursive'', as we obtain the same image contained within itself.

\end{document}  
