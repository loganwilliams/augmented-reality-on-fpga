\documentclass[12pt]{amsart}
\usepackage{geometry}
\geometry{letterpaper}    
\usepackage{graphicx}
\usepackage{amssymb}
\usepackage{epstopdf}
\DeclareGraphicsRule{.tif}{png}{.png}{`convert #1 `dirname #1`/`basename #1 .tif`.png}

\title{Recursive Augmented Reality}
\author{Jos\'{e} E. Cruz Serrall\'{e}s \& Logan Williams}
\date{} 

\begin{document}
\maketitle

In our project, we will implement an augmented reality system that can overlay a digital image on video of a real world environment. We begin by reading NTSC video from a video camera and storing it in a RAM array on the FPGA. We then perform chroma-based object recognition to recognize in the video the corners of a picture frame that have been marked with colored spots. Using the location of these corners, we apply appropriate translation, scaling, rotation, and possibly skew to an image so that it lines up with the edges of the frame. We then output VGA video of the original captured image, with the processed image overlayed on top of the frame. The overlayed image (the "augmentation") can be arbitrary. When this image is the frame of video that was previously displayed, we call the system "recursive", as we obtain the same image within itself. Depending on the precision of our image transformations, we could add many layers to this recursion.

\end{document}  
